% Created 2012-09-20 Thu 13:03
\documentclass[11pt]{article}
\usepackage[utf8]{inputenc}
\usepackage[T1]{fontenc}
\usepackage{fixltx2e}
\usepackage{graphicx}
\usepackage{longtable}
\usepackage{float}
\usepackage{wrapfig}
\usepackage{soul}
\usepackage{textcomp}
\usepackage{marvosym}
\usepackage{wasysym}
\usepackage{latexsym}
\usepackage{amssymb}
\usepackage{hyperref}
\tolerance=1000
\providecommand{\alert}[1]{\textbf{#1}}

\title{Part One: Requirement Specification and Conceptual Design}
\author{Weining Zhang}
\date{\today}
\hypersetup{
  pdfkeywords={},
  pdfsubject={},
  pdfcreator={Emacs Org-mode version 7.8.11}}

\begin{document}

\maketitle


\begin{itemize}
\item Total points: 100
\item Total Weight: 6\%
\item Due: Friday, September 21, 2012, in class.
\item Description
\end{itemize}

In Part One, you will follow the software design methodology of a
database application and carry out conceptual design of a database
application and its conceptual database schema. You should identify an
application domain and a (possibly fictional) organization in this
application domain for which your team is to design and implement a
database application to support its business operations. You need to
describe the business of the organization and its business workflow;
specify the requirements for your database application and your
database; Specify a conceptual design of the application as a set of
functions/procedures and a conceptual database schema as an E/R
diagram and possibly additional constraints. We now discuss specific
issues and requirements.

\section*{Application Domain and organization.}
\label{sec-1}


  You should choose an application domain in which the normal operations
  of an organization typically require to process large set of data in
  non-trivial ways (that is, not only they retrieve data, but they also
  update data). It is not sufficient just to browse the database or just
  to involve a small set of data. Second, you should identify operations
  involved in the business model or workflow of the organization that
  naturally require to collect, update and use data. It is important to
  include activities that cause data to change explicitly or
  implicitly. Third, you should be able to generate or otherwise obtain
  realistic data for the organization you identified.
\section*{Application Program}
\label{sec-2}


  You need to be aware of the complexity of your application program. It
  is not realistic to build a complex real-world application program in
  this project. On the other hand, it is insufficient to develop a
  trivial toy application, either. To design an application with an
  appropriate complexity, you should first identify two to three types
  of users of the application, and then identify several (e.g., 3 to 5)
  types of activities that your program will support for each type of
  users.
\section*{Conceptual Design of Database}
\label{sec-3}


  Your database design should be based on a requirement analysis. You
  should first write a description for each type of users and each type
  of activities. Then, you should create a list of data modeling
  requirements. Next, you will create an E/R diagram of your database
  schema. You may have to adjust your design several times before it is
  finalized with an appropriate complexity. I want to emphasize that a
  good design at beginning of the project is extremely important for the
  success of the entire project. To give you a guideline, I would
  suggest that your E/R diagram contain about 5 entity types and a
  similar number of relationship types. Your specific application may
  require a slight variation to this general estimate. To the extent
  that is appropriate to your specific application, your E/R diagram
  should include various kinds of relationships (one-to-many,
  many-to-many, etc.) and attributes (string, integer, etc.). However,
  using advanced features, such as weak entity type, IS-A relationship,
  etc., is not a requirement.


  The following list of sample application domains may help you start
  thinking about your project. You are encouraged to find your own
  application domain.

  Movie Rental Website: Typical entities in this domain may include
  movies, actors, directors, genres, playing times, reviews, user
  accounts, rental invoice, pickup and return locations and times. There
  exist several sources on the Web where you can find data to populate a
  database. You can support various queries such as finding specific
  playing times, finding movies in San Antonio directed by a given
  director. You can also support updates to movie reviews (e.g., viewers
  giving their own opinions) and rental history. Another functionality
  is to provide personal profiles of people (i.e., the movies they like)
  and then try to recommend movies to them based on profiles of viewers
  with similar tastes.

  On-Line Books Store: In this domain, the entities may include books,
  authors, and topics (which may have a complex hierarchy). You may also
  model various attributes of the authors, the institutions they belong
  to, etc. You can support a buy/sell service of used books, books used
  in specific university courses. A personal profile, similar to the one
  for movies is also a possibility. Pointing an interested buyer to a
  web source to buy a book is also an interesting option.

  Apartment Rental Website: This domain would require the modeling of
  apartments and their properties, areas of town and their various
  properties (e.g., bus lines, crime rate, distance from various
  landmarks). You would provide an interface for offering apartments for
  rent, and finding apartments. Some data can be found from local
  newspapers, realtor web sites, and other sources.

  Social Network Website: In this domain, the basic entities would be
  users, groups, topics, friends, discussion rooms. You can model their
  various properties and relationships. Users can create new groups,
  start discussion on new topics, make new friends, search for users
  with similar interests or in a specific area.
\section*{What to Hand In}
\label{sec-4}


  Hand in a well-formatted written report that includes following items.
\subsection*{Cover Page}
\label{sec-4-1}


   The cover page should have an interesting title of your project, your
   assigned team ID and the names of all team members. The project title
   should clearly identify your application domain and your application
   program. Do not use titles such as ``project 1'' or ``cs3743 project,
   part 1''. Your team ID is assigned after I receive the list of names of
   your teammates.
\subsection*{General Description of Application Domain}
\label{sec-4-2}


   In this section, you should describe your application domain, your
   (fiction or real) organization, the business or operation of the
   organization. This description should be general, informative and
   informal. You may want to point out any unique or especially difficult
   aspect of your application. Your description will be graded on
   content, clarity, and conciseness. (You may want to discuss it with me
   before hand it in).
\subsection*{Conceptual Design of your application Program}
\label{sec-4-3}


   In this section, you should describe proposed functionality of your
   application program. You should describe the types of users of your
   program. The functions that your program will support for different
   types of users. To give you some idea, consider a retail business
   domain. There may be three types of users: customers, salespersons,
   and the shop owner. The application program may provide functions to
   support activities for each type of users. For example, a customer can
   use the program to find product information and prices; and to compare
   products on some predetermined criteria. A salesperson can use the
   program to complete sales transactions, to get customer information,
   to check inventory, etc. A shop owner may use the program to analyze
   sales, to check outstanding orders, and to inspect each business
   transaction. Obviously, your database must contain necessary
   information to support all these activities, and your application
   program needs to implement these supports.
\subsection*{Conceptual Database Schema}
\label{sec-4-4}


   In this section, you provide an E/R diagram of your database. You
   should create a list of data modeling requirements, and translate it
   into an E/R diagram that describe the schema of a database. The E/R
   diagram should be drawn on a single piece of paper and contains
   clearly labeled, meaningful and suitable constraints. Pay special
   attention to follow the convention of E/R model. You should also list
   on a separate piece of paper any business rule that cannot be
   adequately specified in the E/R model.
\subsubsection*{Be aware of following pitfalls in designing E/R diagram.}
\label{sec-4-4-1}


    Design E/R diagram by translating a relational schema. People who
    knows relational database, but are not familiar with E/R model often
    create a relational schema first and translate it into an E/R
    diagram. This will always give you a bad E/R model and you may come up
    a bad design that the E/R model was designed to avoid.  Design an E/R
    diagram to model organization structure. When design E/R diagram based
    on data requirements, some people pay attention to organizational
    structure rather than information needs. If you are not sure, ask
    yourself what operations need to use the entity type or relationship
    type, and why that type of entities or relationships need to be stored
    in the database.  The E/R diagram should contain the entire schema.
    You should maintain a electronic copy of the report, since it is
    likely you will have to revise some part of your design, and you will
    also need to include it in reports of subsequent parts of the
    project. However, you must hand in a hard copy so I can provide you
    feedback.
    

\end{document}