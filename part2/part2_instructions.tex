% Created 2012-10-08 Mon 19:59
\documentclass[11pt]{article}
\usepackage[utf8]{inputenc}
\usepackage[T1]{fontenc}
\usepackage{fixltx2e}
\usepackage{graphicx}
\usepackage{longtable}
\usepackage{float}
\usepackage{wrapfig}
\usepackage{soul}
\usepackage{textcomp}
\usepackage{marvosym}
\usepackage{wasysym}
\usepackage{latexsym}
\usepackage{amssymb}
\usepackage{hyperref}
\tolerance=1000
\usepackage{minted}
\providecommand{\alert}[1]{\textbf{#1}}

\title{Part Two: Logic Design}
\author{Weining Zhang}
\date{\today}
\hypersetup{
  pdfkeywords={},
  pdfsubject={},
  pdfcreator={Emacs Org-mode version 7.8.11}}

\begin{document}

\maketitle


\begin{itemize}
\item Total Points: 100
\item Total Weight: 6\%
\item Due: Friday, October 12, 2012, in class
\end{itemize}

\section*{Description}
\label{sec-1}


In this part, you will complete the logical design of a relational
database for your project. This design consists of two steps. First,
you will translate the E/R diagram obtained in Part One into a
relational schema and then you will normalize each relation schema
into a normal form. Your specific tasks are the following.

\begin{enumerate}
\item Refine E/R diagram.
\end{enumerate}

You should refine your E/R model taken into consideration of my
feedback and your own ideas that may enhance your E/R diagram. You
should draw a new E/R diagram incorporating all the changes. The new
E/R diagram will not be graded, but will be used to grade other tasks
described below.

\begin{enumerate}
\item Translate ER to Relations
\end{enumerate}

You should use the ER-R translation methods discussed in class to
translate your E/R diagram into an initial relational schema.

\begin{enumerate}
\item Specify Functional Dependencies
\end{enumerate}

For each table in the initial relational schema, specify a set of
functional dependencies that would capture the semantics of your
application data. You must identify these functional dependencies
according to your specific application domain. Your primary interest
will be those non-trivial functional dependencies. It is OK if you
could not identify any interesting functional dependency for some
simple tables. Your attention should be focused on tables with more
than two attributes, especially larger ones.

\begin{enumerate}
\item Normalize Tables
\end{enumerate}

For each table in the initial schema, identify its normal form based
on the set of functional dependencies identified in the previous
task. If a relation is not in BCNF, perform the normalization as
described in the class (and in the textbook). Remember to check for
lossless-join and dependency-preserving properties. For each relation,
you should first try to normalize it to BCNF. If failed, you should
then normalize it to 3NF.

\begin{enumerate}
\item Optionally, Combine Tables
\end{enumerate}

Check if you may combine some of the relations without increasing
redundancy. If you can, combine them.
\section*{What to Hand In}
\label{sec-2}


Hand in a well-formatted written report that includes the following
items.

\begin{enumerate}
\item Revised Report of Part One

 This should be a new clean copy that includes revised descriptions of
\end{enumerate}
your application domain and application program, and the revised
ER/EER diagram.

\begin{enumerate}
\item Relational Schema
\end{enumerate}

The final relational schema of your database. For each relation,
include the schema, the list of nontrivial functional dependencies it
satisfies, the list of all candidate keys, the primary key, a list of
all foreign keys and where they reference to, and the normal form that
the relation is in.

\begin{enumerate}
\item Special Note

 If you have combined some relations in the last task listed above,
 include a special note to describe the situation. Also list and
 describe any constraint that your application requires, but can not
 be specified directly in the relational schema.
\end{enumerate}

\end{document}