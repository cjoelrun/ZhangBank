% Created 2012-10-31 Wed 09:29
\documentclass[11pt]{article}
\usepackage[utf8]{inputenc}
\usepackage[T1]{fontenc}
\usepackage{fixltx2e}
\usepackage{graphicx}
\usepackage{longtable}
\usepackage{float}
\usepackage{wrapfig}
\usepackage{soul}
\usepackage{textcomp}
\usepackage{marvosym}
\usepackage{wasysym}
\usepackage{latexsym}
\usepackage{amssymb}
\usepackage{hyperref}
\tolerance=1000
\usepackage{minted}
\providecommand{\alert}[1]{\textbf{#1}}

\title{Part Four: Database Testing}
\author{Weining Zhang}
\date{\today}
\hypersetup{
  pdfkeywords={},
  pdfsubject={},
  pdfcreator={Emacs Org-mode version 7.8.11}}

\begin{document}

\maketitle


\begin{itemize}
\item Total Points: 100
\item Total Weight: 6\%
\item Due: Friday, November 9, 2012, in class
\end{itemize}

\section*{Description}
\label{sec-1}


  In this part, you will use Oracle to test and to manipulate your
  database. The purpose are two-folds. First, to design SQL queries
  that will eventually be included in the implementation of your
  application program. Second, to make sure the data in your database
  is sufficient for testing your application. Specifically, you need
  to do the following.

\begin{enumerate}
\item Learn SQL*PLUS

     You should try SQL*PLUS commands and SQL statements
     interactively, run SQL statements from a script file, and edit
     and run commands from the command buffer.
\item Test SQL Queries

     You should write and test at least 10 interesting, non-trivial
     SQL queries on your database. Ideally, the queries should include
     the ones you plan to use in your application program to implement
     services for users. Nevertheless, you should include at least one
     query in each of the following query types. Of course, some of
     your queries may be in more than one type.
\begin{enumerate}
\item Multiple table query;
\item Nested query (both correlated and not correlated);
\item Query using Union, Intersect, and minus;
\item Query using exist, not exist, IN, NOT IN, ALL, etc.;
\item Query with aggregate functions, group by, sort by, and having;
\item Query that has a complex from clause; and
\item Query using a view.
\end{enumerate}
\item Database Update
     
     You should use SQL to make at least two updates of each of the
     following types:
\begin{enumerate}
\item Insert tuples to a table (both single tuple and by a query);
\item Delete tuples from a table (both single tuple and by a query);
\item Updates that changes existing tuples (both single tuple and by
        a query); and
\item Updates that cause a trigger to be fired.
\end{enumerate}
\item Additional Schema Objects

     In addition to what you already have, you will add some new
     features to your database and test these features. More
     specifically, you need to do the following.
\begin{enumerate}
\item Design at least one view that accesses two or more base tables.
\item Design and test at least one trigger.
\item Design and perform some experiments to investigate view
        update. These experiments should try to perform various types
        of update on various type of views, and investigate if and how
        data can be updated through using views.
\end{enumerate}
\end{enumerate}

  While you are doing this part, keep in mind the following:
  
\begin{itemize}
\item Test queries that can be used in your application program. Think
    how to retrieve data to provide services of your application
    program.
\item Specify each query in different ways. It is well known that a
    query can be expressed in many ways in SQL. Pay attention to
    whether a result contains duplicate.
\item Make sure that your interesting non-trivial queries get non-empty
    answers. Add more data to your database if necessary.
\item Come up with each query in English first and then translate it
    into SQL.
\end{itemize}
\section*{What to Hand In}
\label{sec-2}


Hand in a well-formatted written report that includes the following
items.

\begin{enumerate}
\item Revised report of Part Three

   This should be a fresh copy of Part Three incorporating all changes
   to all previous parts.
\item SQL Queries
   
   A list of your SQL queries. For each query, list the English
   version followed immediately by the SQL statements. These queries
   will be graded based on if they are interesting and if they are
   correct with respect to their English statements.
\item Views and Triggers

   A list of SQL statements that define your view and trigger.
\item Spool File

   A spool file that illustrates a successful execution of your
   queries, updates, views and triggers. Do not include answers that
   contain many pages of data. For updates, show the relevant data
   before and after the execution of your update commands.
\item View Investigation
   
   Include a two-page report describing your experiments with view
   update including the design of the experiments, your observations,
   and your conclusions. This report will be graded based on the
   design of experiments and on the quality of the technical writing.
\end{enumerate}

\end{document}