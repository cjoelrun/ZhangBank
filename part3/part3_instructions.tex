% Created 2012-10-22 Mon 15:01
\documentclass[11pt]{article}
\usepackage[utf8]{inputenc}
\usepackage[T1]{fontenc}
\usepackage{fixltx2e}
\usepackage{graphicx}
\usepackage{longtable}
\usepackage{float}
\usepackage{wrapfig}
\usepackage{soul}
\usepackage{textcomp}
\usepackage{marvosym}
\usepackage{wasysym}
\usepackage{latexsym}
\usepackage{amssymb}
\usepackage{hyperref}
\tolerance=1000
\usepackage{minted}
\providecommand{\alert}[1]{\textbf{#1}}

\title{Part Three: Database Creation}
\author{Weining Zhang}
\date{\today}
\hypersetup{
  pdfkeywords={},
  pdfsubject={},
  pdfcreator={Emacs Org-mode version 7.8.11}}

\begin{document}

\maketitle


\begin{itemize}
\item Total Points: 100
\item Total Weight: 6\%
\item Due: Friday, October 26, 2012, in class
\end{itemize}

\section*{Description}
\label{sec-1}


  In this part, you will create a relational database for your
  application. This includes the creation of your database in Oracle
  and the loading of data into the database. Specifically, your need
  to do the following.

\begin{enumerate}
\item Learn to Use Oracle

     You need to get familiar with Oracle SQL*PLUS environment. You
     may want to login to SQL*PLUS, learn to use various commands, try
     out online help, and read on-line documents.
\item Refine Relational Schema
     
     Before creating the database, you need to refine the relational
     schema of your database based on my feedback on your Part Two of
     the project.
\item Define schema in Oracle 

     You will convert the refined relational schema into SQL
     statements that create tables. You need to decide SQL types for
     attributes and constraints for tables and for columns. Here are
     some specific requirements.
\begin{enumerate}
\item You should use various Oracle data types in your
       tables. Specifically, you should have at least one attribute in
       each of the following types: integer, real, character string of
       fixed length, character string of a variable length, date,
       sequence, and enumerated values (using a set).
\item Both table constraints and column constraints should be
       specified. In addition, you should define at least one object
       type, and one CHECK constraint.
\item You should specify primary and foreign keys.

       You should keep these SQL DDL statements in a .sql script file,
       and run it from within SQL*PLUS to create tables.
\end{enumerate}
\item Load data to your database

     You should use Oracle SQL*LOADER to populate tables with your
     application data. You should provide enough data so that complex
     queries will not always result in empty answers. As a guideline,
     each table should have about 40 to 50 tuples. For those tables
     that likely to be involved in multi-table SQL queries for your
     application, make sure that tuples in different tables will
     actually join. You do not need to worry about tables that are
     inherently small. You do not have to make tables really big at
     this time, you can always add more data in future.
\end{enumerate}
\section*{What to Hand In}
\label{sec-2}


  Hand in a hard copy report that contains the following items.

\begin{enumerate}
\item Revised Report of Part Two
     
     Again, include a fresh copy of the revised report of Part
     Two. Make sure to include reports for Part One and Two with all
     the revisions. (So that the current report gives a snapshot of
     the current status of the project).
\item SQL Scripts 

     Include a hard copy of SQL script files that you use to create
     domains, tables, views, and sequences, and also SQL*LOADER
     control files that you use to load your data. (Do not include the
     data file.)
\item Spool File
     
     Include a spool file that illustrate a session in which you
     successfully created your database and loaded the data into your
     database. Please do not hand in pages after pages of data.
\item Special Note
     
     Include a special note to describe any business rule in the
     application domain that can not be directly specified in the SQL
     database schema.
\end{enumerate}

\end{document}